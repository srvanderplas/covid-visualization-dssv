\documentclass[article]{jdssv}\usepackage[]{graphicx}\usepackage[]{color}
% maxwidth is the original width if it is less than linewidth
% otherwise use linewidth (to make sure the graphics do not exceed the margin)
\makeatletter
\def\maxwidth{ %
  \ifdim\Gin@nat@width>\linewidth
    \linewidth
  \else
    \Gin@nat@width
  \fi
}
\makeatother

\definecolor{fgcolor}{rgb}{0.345, 0.345, 0.345}
\newcommand{\hlnum}[1]{\textcolor[rgb]{0.686,0.059,0.569}{#1}}%
\newcommand{\hlstr}[1]{\textcolor[rgb]{0.192,0.494,0.8}{#1}}%
\newcommand{\hlcom}[1]{\textcolor[rgb]{0.678,0.584,0.686}{\textit{#1}}}%
\newcommand{\hlopt}[1]{\textcolor[rgb]{0,0,0}{#1}}%
\newcommand{\hlstd}[1]{\textcolor[rgb]{0.345,0.345,0.345}{#1}}%
\newcommand{\hlkwa}[1]{\textcolor[rgb]{0.161,0.373,0.58}{\textbf{#1}}}%
\newcommand{\hlkwb}[1]{\textcolor[rgb]{0.69,0.353,0.396}{#1}}%
\newcommand{\hlkwc}[1]{\textcolor[rgb]{0.333,0.667,0.333}{#1}}%
\newcommand{\hlkwd}[1]{\textcolor[rgb]{0.737,0.353,0.396}{\textbf{#1}}}%
\let\hlipl\hlkwb

\usepackage{framed}
\makeatletter
\newenvironment{kframe}{%
 \def\at@end@of@kframe{}%
 \ifinner\ifhmode%
  \def\at@end@of@kframe{\end{minipage}}%
  \begin{minipage}{\columnwidth}%
 \fi\fi%
 \def\FrameCommand##1{\hskip\@totalleftmargin \hskip-\fboxsep
 \colorbox{shadecolor}{##1}\hskip-\fboxsep
     % There is no \\@totalrightmargin, so:
     \hskip-\linewidth \hskip-\@totalleftmargin \hskip\columnwidth}%
 \MakeFramed {\advance\hsize-\width
   \@totalleftmargin\z@ \linewidth\hsize
   \@setminipage}}%
 {\par\unskip\endMakeFramed%
 \at@end@of@kframe}
\makeatother

\definecolor{shadecolor}{rgb}{.97, .97, .97}
\definecolor{messagecolor}{rgb}{0, 0, 0}
\definecolor{warningcolor}{rgb}{1, 0, 1}
\definecolor{errorcolor}{rgb}{1, 0, 0}
\newenvironment{knitrout}{}{} % an empty environment to be redefined in TeX

\usepackage{alltt}

%% -- LaTeX packages and custom commands ---------------------------------------

%% recommended packages
\usepackage{thumbpdf,lmodern}
\usepackage{bm}
\usepackage{todonotes} %% Remove this package when finishing the manuscript


%% another package (only for this demo article)
\usepackage{framed}

\usepackage{cleveref}

%% new custom commands
\newcommand{\class}[1]{`\code{#1}'}
\newcommand{\fct}[1]{\code{#1()}}
\newcommand{\ma}[1]{\ensuremath{\mathbf{#1}}}


%% -- Article metainformation (author, title, ...) -----------------------------

%% - \author{} with primary affiliation
%% - \Plainauthor{} without affiliations
%% - Separate authors by \And or \AND (in \author) or by comma (in \Plainauthor).
%% - \AND starts a new line, \And does not.
\author{Susan Vanderplas\\University of Nebraska–Lincoln
   \And Adalbert F.X. Wilhelm\\Jacobs University Bremen}
\Plainauthor{Susan Vanderplas, Adalbert F.X. Wilhelm}

%% - \title{} in title case
%% - \Plaintitle{} without LaTeX markup (if any)
%% - \Shorttitle{} with LaTeX markup (if any), used as running title
\title{Visual narratives of the Covid-19 pandemic}
\Plaintitle{Visual narratives of the Covid-19 pandemic}
\Shorttitle{Visual narratives of Covid-19}

%% - \Abstract{} almost as usual
\Abstract{
  Covid-19 has sparked a worldwide interest in understanding the dynamic evolution of a pandemic and tracking the effectiveness of preventive measures and rules. For this reason, numerous media and research groups have produced comprehensive data visualisations to illustrate the relevant trends and figures. In this paper, we will look at a selection of Covid 19 data visualisations to evaluate and discuss the currently established visualisation tools in terms of their ability to provide a communication channel both within the data science team and between data analysts, domain experts and a general interested audience. Although there is no set catalogue of evaluation criteria for data visualisations, we will try to give an overview of the different core aspects of visualisation evaluation and their competing principles.
  }

%% - \Keywords{} with LaTeX markup, at least one required
%% - \Plainkeywords{} without LaTeX markup (if necessary)
%% - Should be comma-separated and in sentence case.
\Keywords{exploratory data visualisation, logarithmic scales, visual comparisons, \proglang{R}}
\Plainkeywords{exploratory data visualisation, logarithmic scales, visual comparisons, R}

%% - \Address{} of at least one author
%% - May contain multiple affiliations for each author
%%   (in extra lines, separated by \emph{and}\\).
%% - May contain multiple authors for the same affiliation
%%   (in the same first line, separated by comma).
\Address{
    email: \email{susan.vanderplas@unl.edu}
  Susan Vanderplas\\
  Department of Statistics\\
  University of Nebraska–Lincoln\\
  349A Hardin Hall \\
  Lincoln, NE 68583-0963, USA\\
  E-mail: \email{susan.vanderplas@unl.edu}\\
  URL: \url{https://statistics.unl.edu/susan-vanderplas}\\
\ \\
%}
%
%\Address{
  Adalbert F.X. Wilhelm\\
  Department of Psychology and Methods\\
  Jacobs University Bremen gGmbH\\
  Campus Ring 1\\
  28759 Bremen, Germany\\
  E-mail: \email{a.wilhelm@jacobs-university.de}\\
  URL: \url{http://www.jacobs-university.de/directory/wilhelm}
}
\IfFileExists{upquote.sty}{\usepackage{upquote}}{}
\begin{document}
% \SweaveOpts{concordance=TRUE}





%% -- Introduction -------------------------------------------------------------

%% - In principle "as usual".
%% - But should typically have some discussion of both _software_ and _methods_.
%% - Use \proglang{}, \pkg{}, and \code{} markup throughout the manuscript.
%% - If such markup is in (sub)section titles, a plain text version has to be
%%   added as well.
%% - All software mentioned should be properly \cite-d.
%% - All abbreviations should be introduced.
%% - Unless the expansions of abbreviations are proper names (like "Journal
%%   of Statistical Software" above) they should be in sentence case (like
%%   "generalized linear models" below).

\section{Introduction}

Over the past two years, several waves of Covid-19 infections with different mutants of the SARS-CoV-2 virus have swept across the globe, claiming many lives, causing numerous health damages and affecting our personal lives in many ways. According to the WHO, over 304 million confirmed cases and over 5.4 million deaths have been reported as of early January 2022 \footnote{\url{https://www.who.int/publications/m/item/weekly-epidemiological-update-on-covid-19---11-january-2022}}. The pandemic has generated enormous interest in epidemiological data, its analysis and visualisation. From the beginning of the pandemic, data on the number of infections and covid-related deaths has been published daily and made available to the public. Media, politicians and individuals use this data to build their narratives about the pandemic, discuss its evolution, justify the measures taken and discuss different prevention strategies against the spread of the virus. So there are several goals to be achieved by visualising Covid-19 data. These goals and their priorities were adapted as dynamically as the virus mutated and the pandemic changed pace. But as with any other data visualisation two general principles remain the same: ensuring clear vision by optimising the data-to-ink ratio \citep{Tufte2001} (or signal-to noise ratio) and ensuring clear understanding by organising the graphics in such a way that the story of the data is told most effectively.

In recent decades, many media organisations have established data teams that received unprecedented attention and wide-ranging opportunities during the pandemic as they have been showcasing their skills and abilities, not only in visualising data, but also in explaining their data collection and data analysis strategies and methods. Data journalism will certainly be one of the beneficiaries of the Covid-19 pandemic and it has become an innovative part of news publishing, with COVID-19 delivering many excellent applications, often presented in an interactive visual format on the web, such as dashboards.

%shall we include a list of newspapers/media outlets that we aim at covering?

At the same time we still see a lot of defective graphics disseminated and shared: some that violate fundamental visualisation and statistical reporting principles such as accuracy, relevance, timeliness, clarity, coherence, and reproducibility. These principles have been laid out in numerous standards for statistical reporting in the application areas, such as the ESS standard for quality reports \citep{ess2009}, the CONSORT, PRISMA, CHEERs guidelines, and others (see https://equator-network.org). Numerous publications, initiatives, and ideas to improve the communication of quantitative and statistical information have been prepared, see for example \cite{Hoffrage2261,Tufte2001,Rosling2011,otavamylona2020}.

The Covid 19 pandemic has amply demonstrated that policies are only effectively implemented and followed by the population if they are accepted by a large majority of the population. To achieve this goal, effective communication of quantitative evidence and statistical results between scientists, governments, the media and citizens is essential to justify the appropriateness, usefulness and relevance of the measures. 

A best practice for effective communication is to present information in an understandable way (Gigerenzer et al., 2007; Gigerenzer and Edwards, 2011), for example by saying "one in ten" instead of 10\%. Using absolute rather than relative numbers, presenting information in an appealing graphic form and summarising the most important facts in "fact boxes" are some of the methods that have been developed and advocated to increase transparency in communication between stakeholders such as healthcare providers and patients (see for example: https:// www.hardingcenter.de).
Fact boxes in combination with icon arrays are recommended for the presentation of test results. Both representations are based on natural frequencies \citep{[62],Krauss2020} and represent case numbers as simply and concretely as possible. Many scientific studies show that icon arrays help people to understand numbers and risks more easily (e.g. \cite{McDowell2019}). The Harding Center for Risk Literacy shows many other examples of transparent communication of risks, including COVID-19\footnote{\url{https://www.hardingcenter.de/de/mrna-schutzimpfung-gegen-covid-19-fuer-aeltere-menschen}}.

Nevertheless, both the complexity of phenomena and the "bipolarity" of statistical thinking remain a challenge. 

While human thinking tends to simplify patterns and political communication also prefers a simple cause-effect relationship, real phenomena are often multivariate. Thus, in studying COVID-19 and predicting its spread, it is not only important to consider symptomatology, disease incidence and geographic distribution, population behaviour patterns, government policies, and impacts on the economy, schools, people in nursing homes, and society at large, but also to incorporate these into data analyses and communication of results. Associations observed in the data can often be caused by third party variables (confounders). In addition, much of the data comes from observational studies, which usually makes robust causal attribution problematic. However, statisticians who point out these limitations are at risk of having their statements pulled out one-sidedly in a polarised debate \citep{McConway2021}.

\section[]{The global perspective}

On 11 March 2020, WHO declared the outbreak of the novel coronavirus disease (COVID-19) a pandemic, and since that date at the latest, the global perspective of the disease has been in the public eye. The spatial spread of the virus and the resulting cases and deaths are commonly visualized by choropleth maps, see for example Figure~\ref{fig:choro1} showing the total number of infections reported in each country as of January 14, 2022. For the pandemic perspective a central element of the narrative is the ubiquity of the disease and the accompanying global impact. Choropleth maps based on raw numbers of cases might look convincing and fit to the purpose, but neglect a number of well-know caveats for statistical reporting and visualisation:
\begin{enumerate}
\item Absolute value unsuitability: As explained in \citep{monmonier2005, slocum2008, speckmann2010} among others, choropleth maps are fundamentally unsuitable for the representation of absolute numbers. Especially in the case of similarly coloured areas of the regions, viewers tend to integrate them unconsciously and perceive choropleths as representations of density. They also do not help to convey the desired message as the absolute numbers of Covid-19 cases are strongly influenced by the population size of the country, but also by the number of tests performed and the accuracy of the recording and reporting system.
\item The area-bias: The visual impression is determined more by the colour and the geographical area of the individual countries than by the number of Covid cases. Since the countries of the world differ extremely in area, the visual assessment is distorted, especially in the case of neighbouring countries with similar numbers but different areas.
\item Color-scheme obstructions: Much research in visualisation is concerned with the appropriate choice of colour schemes, see \citep{brewer1997}. The choice of a continuous scale or a categorical scale, the choice of a scale that promotes the recognition of patterns or a scale that supports the filtering out of specific map details, influences the quality of choropleth maps.  
\end{enumerate}

\begin{figure*}
	\includegraphics[width = 0.98\textwidth]{Figures_Web/who_totalcases_choro.png}
	\caption{Choropleth map of the Covid-19 cases world-wide by country. Source: WHO \url{https://covid19.who.int}}
	\label{fig:choro1}
\end{figure*}

Choropleth maps are quite commonly used by media companies and governmental organisations and it is easy to find good and bad examples of their usage.

Proportional symbol maps, see Fig.~\ref{fig:propsymb} or graduated symbol maps place scaled symbols or diagrams directly on the input map, often on the centroid of the regions. The symbol, most commonly a disk or a square, is scaled such that its area corresponds to the data value of the region.


\begin{figure*}
	\includegraphics[width = 0.98\textwidth]{Figures_Web/wp_totaldeaths_propsymb.png}
	\caption{Proportional symbol map showing the cumulated reported Covid-19 related deaths adjusted for population world-wide by country. Source: Washington Post \url{https://www.washingtonpost.com/graphics/2020/world/mapping-spread-new-coronavirus/}}
	\label{fig:propsymb}
\end{figure*}

And clearly, the visual impression of any map based illustration depends on the projection chosen for the underlying map as the examples quite often exercies a western-centric viewpoint.

The narrative of the global perspective seems to be highly limited to the aspect of a pandemic affecting the entire globe. None of the above viusalisations intends to provide a deeper insight into the spatial distribution of the phenomenon, neither within the administratively motivated spatial borders nor across them. The use of maps in this context more often focusses on the comparative aspects: how are we doing as opposed to our neighbors, which startegy to fight Covid-19 is better? Topics that we will look closer into in Section~\ref{sec:rankings}

Another illustrative example is a simulation from ZEIT Online\footnote{\url{https://www.zeit.de/wissen/gesundheit/2020-11/coronavirus-aerosole-ansteckungsgefahr-infektion-hotspot-innenraeume}}, which - based on models developed by a group of researchers at the Max Planck Institute for Chemistry - estimates the probability of an infected person infecting other people in closed rooms in various scenarios. While the visualisations present the simulated infection processes in a catchy way, the dependence of the simulations on parameter assumptions and settings is usually not addressed. Simulations should also always make transparent on which model assumptions and which data basis the simulations were created.


\section{Time series and dynamics of the outbreak}
% Line plots, frequency domain, NY times tile plots w/ cases represented in color intensity




Simulations were used particularly illustratively in the media in the course of the pandemic. An inspiring example illustrating the spread of the epidemic appeared as early as 14 March 2020 in the Washington Post\footnote{\url{https://www.washingtonpost.com/graphics/2020/world/corona-simulator/}} with the title "Why outbreaks like coronavirus spread exponentially, and how to flatten the curve". The Washington Post made this simulation available free of charge and in all major languages, which led to it being distributed worldwide, including repeatedly on German television\footnote{\url{https://web.br.de/interaktiv/corona-simulation/}}. The New York Times\footnote{\url{https://www.nytimes.com/interactive/2020/us/coronavirus-spread.html}} published a dynamic graphic entitled "How the Virus Won", which maps the spread of COVID-19 cases from February to June 2020 in the USA. It shows how an analysis of the associations between different COVID-19 strains and travel patterns can help understand the spread of the disease.



\section[]{Comparisons and rankings}
\label{sec:rankings}

The plague is always the others: This is the short formula for dealing with infectious diseases from a historical perspective \citep{thiessen2021}. Sociologists have coined the term "othering" for such attributions of the others \citep{mountz2009}. This refers to the observation that above all new, unknown threats are projected onto "strangers" and "the others". Closing the borders and restricting access to the country became a popular means of controlling the spread of the disease. Unsurprisingly, the prevailing visual narrative focused on comparisons often fuelled by political rivalry, historical dependencies, recent withdrawal from supranational institutions or regional competitions. An ongoing debate about the true extent of the dangers of covid-19 and how best to combat it, combined with daily availability and public access to data at all types of adminsitratory levels, fostered ongoing competition and the derivation of leaderboards. 
% https://twitter.com/AndToddsaid/status/1314231055096971265 is a horrible set of graphs and a great example of the snark associated with covid management policy and geographic comparisons... 



\section{To log or not to log}
As COVID cases grow quasi-exponentially while there are susceptible members of the population (subject to the effectiveness of mitigation measures and testing availability), it seems natural to use log scales to allow for more effective comparisons of slight changes in case counts over time. In addition, log scales make it possible to compare regions with different populations or infection rates in the same chart. As noted previously, however, interpreting log scales requires levels of numerical sophistication that may not be appropriate for the general public. Even researchers do not always read and interpret log scales correctly\citep{mengeLogarithmicScalesEcological2018}; expecting the general public to do so is difficult under normal circumstances\citep{hecklerStudentAccuracyReading2013} is difficult. When panic, fear, uncertainty, and doubt about the situation are added to the mix, 
% Discussion of how this section's conclusions change with the pandemic

One issue with assessing the use of log scales is that their effectiveness changes with the stage of the pandemic and the amount (and varieties) of data shown. Initially, log scales were incredibly useful at showing case counts, because minimal mitigation measures were in place and the growth of case counts (or presumptive positive cases, in absence of available testing) was fairly close to exponential. In addition, the use of log scales allowed for the comparison of nominal cases across entities with large population differences: in the US, we could compare cases in New York and California with cases in Michigan and Washington, even though the population of Michigan and Washington are much lower than the population of either New York or California. 

\begin{knitrout}\footnotesize
\definecolor{shadecolor}{rgb}{0.961, 0.961, 0.961}\color{fgcolor}\begin{figure}

{\centering \includegraphics[width=\linewidth]{Figures_R/fig-log-scale-initial-1} 

}

\caption[In the early stages of the pandemic, log scales allowed the comparison of raw case counts in locations with vastly different population and case counts]{In the early stages of the pandemic, log scales allowed the comparison of raw case counts in locations with vastly different population and case counts.}\label{fig:log-scale-initial}
\end{figure}

\end{knitrout}

While log scales are not necessarily intuitive, many outlets tried to make the graphs more intuitive by adding reference lines, as shown in \Cref{fig:diag-ref-lines}. 

\begin{figure}
\centering
\includegraphics[width=.8\linewidth]{ft-covid-19-deaths}
\caption{Reference lines to compare exponential growth rates of deaths in different countries. This provides some additional context that may help individuals use log scale data more successfully. This approach was first featured in the Financial Times, but was quickly adopted by the New York Times, 91-DIVOC, and other outlets. Graph from the Financial Times (March 23, 2020), image from \citet{kosaraPraiseDiagonalReference2020}.}
\label{fig:diag-ref-lines}
\end{figure}



% Problems with log scales
However, after the first wave of COVID, the issues with log scales became more apparent: it was difficult to detect slight increases in case counts that indicated the beginning of a new wave amid a background level of spread, as demonstrated in  \Cref{fig:log-scale-failures}. Diagonal reference lines from the origin were also less helpful, as the growth of cases or deaths was no longer approximately exponential and varied over time; for these reference lines to be effective there would need to be a clear idea of when the case counts started to increase exponentially, which is difficult to determine whilst in the thick of a potential COVID wave. 

\begin{knitrout}\footnotesize
\definecolor{shadecolor}{rgb}{0.961, 0.961, 0.961}\color{fgcolor}\begin{figure}

{\centering \includegraphics[width=.95\linewidth]{Figures_R/fig-log-scale-failures-1} 

}

\caption[One problem with log scales is that if there is a background level of spread, it can be hard to notice the introduction of an additional source of exponential spread]{One problem with log scales is that if there is a background level of spread, it can be hard to notice the introduction of an additional source of exponential spread. Linear scales do not have this problem - the exponential source is noticeable very quickly in the total line, but on the log scale it is much harder to discern when the exponential source causes the total line to diverge from the background. In the top-right corner, it is difficult to identify that there is an exponential increase in cases amid the baseline, even though the exponential source makes up approximately 50\% of the cases at the end of the time period shown.}\label{fig:log-scale-failures}
\end{figure}

\end{knitrout}

% Focus primarily on time-series graphs (as opposed to log scales in color)


% Problems with linear scales
While log scales have their problems, linear scales are not immune from issues either. It can be very difficult to adequately compare to past situations when looking at the full time series of case counts. For example, in \Cref{fig:linear-scales-ref-lines}, it is difficult to tell whether the first wave of COVID cases in March 2020 had an increase as fast as that in January of 2021; it is even more difficult to compare the order-of-magnitude of change in case rate growth of January 2021 relative to January 2022 when the more contagious omicron variant became prevalent. 

\begin{knitrout}\footnotesize
\definecolor{shadecolor}{rgb}{0.961, 0.961, 0.961}\color{fgcolor}\begin{figure}

{\centering \includegraphics[width=.8\linewidth]{Figures_R/fig-linear-scales-ref-lines-1} 

}

\caption[Reported COVID cases in New York State, 2020-2022]{Reported COVID cases in New York State, 2020-2022. The linear scale makes it difficult to compare the trajectory of different waves to determine how severe the current status is relative to the past, because the primary contrast is the height of the relative peaks, rather than the growth \emph{rate}. A similar graph on the log scale would have the peaks at much more similar heights (though there would still be a difference), allowing the reader to focus on other information.}\label{fig:linear-scales-ref-lines}
\end{figure}

\end{knitrout}


\begin{knitrout}\footnotesize
\definecolor{shadecolor}{rgb}{0.961, 0.961, 0.961}\color{fgcolor}\begin{figure}

{\centering \includegraphics[width=.8\linewidth]{Figures_R/fig-log-scales-ref-lines-1} 

}

\caption[Reported COVID cases in New York State, 2020-2022]{Reported COVID cases in New York State, 2020-2022. The linear scale makes it difficult to compare the trajectory of different waves to determine how severe the current status is relative to the past.}\label{fig:log-scales-ref-lines}
\end{figure}

\end{knitrout}



% Discussion of sites which allow the user to switch back and forth between scales

It is not clear that the use of log or linear scales during the COVID-19 pandemic had a large effect on public opinion. Several studies were conducted in the early stages of the pandemic \citep{romanoScaleCOVID19Graphs2020, seviLogarithmicLinearVisualizations2020, ryanGraphsLogarithmicAxes2020} and results seem to suggest that while individuals have difficulty understanding log scale graphs, these issues do not tend to affect their support for intervention measures (perhaps in part because COVID-related news saturated the news and opinions were set outside of information provided in the experiments). 


\section{Summary and discussion} \label{sec:summary}


%% -- Optional special unnumbered sections -------------------------------------
\newpage
\section*{Computational Details}

If necessary or useful, information about certain computational details
such as version numbers, operating systems, or compilers could be included
in an unnumbered section. Also, auxiliary packages (say, for visualizations,
maps, tables, \dots) that are not cited in the main text can be credited here.


The results in this paper were obtained using
\proglang{R}~3.5.1. \proglang{R} itself
and all packages used are available from the Comprehensive
\proglang{R} Archive Network (CRAN) at
\url{https://CRAN.R-project.org/}.


\section*{Acknowledgments}

All acknowledgments should be collected in this
unnumbered section before the references. It may contain the usual information
about funding and feedback from colleagues/reviewers/etc. Furthermore,
information such as relative contributions of the authors may be added here
(if any).


%% -- Bibliography -------------------------------------------------------------
%% - References need to be provided in a .bib BibTeX database.
%% - All references should be made with \cite, \citet, \citep, \citealp etc.
%%   (and never hard-coded). See the FAQ for details.
%% - JSS-specific markup (\proglang, \pkg, \code) should be used in the .bib.
%% - Titles in the .bib should be in title case.
%% - DOIs should be included where available.

\bibliography{refs}


%% -- Appendix (if any) --------------------------------------------------------
%% - After the bibliography with page break.
%% - With proper section titles and _not_ just "Appendix".

\newpage

\begin{appendix}

\section{More Technical Details} \label{app:technical}


Appendices can be included after the bibliography (with a page break). Each section within the appendix should have a proper section title (rather than just \emph{Appendix}).

%For more technical style details, please check out JSS's style FAQ at
%\url{https://www.jstatsoft.org/pages/view/style#frequently-asked-questions}
%which includes the following topics:
%\begin{itemize}
%  \item All main words in titles and headers start with a capital.
%  \item Use vectorized graphics formats such as eps and pdf when possible. Graphics should be of high quality while trying to minimize the file size.
%\end{itemize}


\section[Using BibTeX]{Using \textsc{Bib}{\TeX}} \label{app:bibtex}

References need to be provided in a \textsc{Bib}{\TeX} file (\code{.bib}). All references should be made with \verb|\cite|, \verb|\citet|, \verb|\citep|, \verb|\citealp| etc.\ (and never hard-coded). This commands yield different formats of author-year citations and allow to include additional details (e.g., pages, chapters, \dots) in brackets. 
%In case you are not familiar with these commands see the JSS style FAQ for details.

Cleaning up \textsc{Bib}{\TeX} files is a somewhat tedious task -- especially when acquiring the entries automatically from mixed online sources. However, it is important that informations are complete and presented in a consistent style to avoid confusions. JDSSV requires the following format.
\begin{itemize}
  \item Specific markup (\verb|\proglang|, \verb|\pkg|, \verb|\code|) should
    be used in the references.
  \item Titles should be inserted in title case.
  \item Journal titles should not be abbreviated and in title case.
  \item DOIs should be included where available.
  \item Software should be properly cited as well. For \proglang{R} packages
    \code{citation("pkgname")} typically provides a good starting point.
\end{itemize}
\end{appendix}
\newpage
%% -----------------------------------------------------------------------------


\end{document}
